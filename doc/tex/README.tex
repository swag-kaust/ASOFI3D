\subsection{Obtaining the code}\label{obtaining-the-code}

Get the code from this repo:

\begin{verbatim}
git clone git@github.com:swag-kaust/ASOFI3D.git
\end{verbatim}

Then switch to the created directory:

\begin{verbatim}
cd ASOFI3D
\end{verbatim}

\subsection{Building the code}\label{building-the-code}

The prerequisites for ASOFI3D are:

\begin{itemize}
\tightlist
\item
  C compiler (for example, \texttt{gcc}), which supports C11 standard
\item
  MPI library (for example, \href{https://www.open-mpi.org/}{OpenMPI})
\item
  \href{https://www.gnu.org/software/make/}{GNU Make} build system
\end{itemize}

\subsection{Building the code with gcc and OpenMPI on
Ubuntu}\label{building-the-code-with-gcc-and-openmpi-on-ubuntu}

On recent Ubuntu versions such as 14.04, 16.04, or 18.04 all
prerequisites can be obtained by the following commands:

\begin{verbatim}
sudo apt-get install gcc libopenmpi-dev make
\end{verbatim}

Then while in the root directory of the code, build the code via command

\begin{verbatim}
make
\end{verbatim}

which compiles the solver and several auxiliary utilities.

Before compilation, make will automatically test for the ability to
compile MPI programs and generate file \texttt{src/config-auto.mk} with
the flags for compilation.

\subsection{Example usage}\label{example-usage}

After successful compilation, you can run the code via command

\begin{verbatim}
./run_asofi3D.sh np dirname
\end{verbatim}

where \texttt{np} is a number of MPI processes you want to use and
\texttt{dirname} is the directory that contains configuration of the
problem to solve. Parameter \texttt{dirname} is optional and defaults to
\texttt{par}, so that the main configuration file of the solver is
\texttt{par/in\_and\_out/asofi3D.json}.

\subsection{Running the tests}\label{running-the-tests}

To run the tests, \href{http://www.ahay.org}{Madagascar} is an
additional prerequisite. Tests are run via the command

\begin{verbatim}
make test
\end{verbatim}
